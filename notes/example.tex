\documentclass{article}
\usepackage{notes}


%\date{28 March 2010} % without \date command, current date is supplied

\title{A Style for Random Notes}
\author{Shawn O'Hare}

\begin{document}

\maketitle

\begin{abstract}
\noindent
This document serves an example utilizing the styles provided by
notes.sty
\end{abstract}

\tableofcontents
%\printclassoptions

\section{Page Layout}\label{sec:page-layout}
\subsection{Headings}\label{sec:headings}

% let's start a new thought -- a new section
In his later books,~\cite{Tufte2006} Tufte
starts each section with a bit of vertical space, a non-indented paragraph,
and sets the first few words of the sentence in \textsc{small caps}.

\begin{definition}[Gadget]
  A \emph{gadget} is a thing.
  Each gadget comes equipped with some data.
\end{definition}

Various special types of gadgets exist.  For example, the widget is
commonly encountered in practice.

\begin{definition}[widget]
  A \emph{widget} is a type of gadget.
\end{definition}

Below we consider some examples of gadgets.  This text serves as an example of a
paragraph inbetween two theorem-like environments.

Another paragraph to demonstrate space rendering between paragraphs. Sometimes a
line break between each paragraph is desirable.

\begin{example}[Gadget]
  A set is a type of gadget.
\end{example}

This paragraph serves as some text before a remark, which in the
amsthm style has no extra space above or below.

\begin{remark}[Some remark]
  Now, let's consider some remark.
\end{remark}

This paragraph serves as some text after a remark, which in the
amsthm style has no extra space above or below.


\begin{theorem}[Quotient Groups]
  Let $G$ be a finite group and $H$ a normal subgroup.
  Then the left cosets $G / H$ form a group under
  left multiplication via $(xH)(yH) = xyH$.
\end{theorem}

\begin{proof}
  Since $H$ is normal in $G$, for any $x \in G$
  $H =  x^{-1}Hx$. Multiply by $x$ on both sides yields
  $xH=Hx$. It follows for any
  $x$, $y$ in $G$ that
  \begin{align}
    (xH)(yH)
    &= x(Hy)H \\
    &= x(yH)H \\
    &= xy(HH) \\
    &=xyH.
  \end{align}
  From this it easily follows that $G / H$ is a group.
\end{proof}

\begin{table}\label{tab:1}
  \centering
  \begin{tabular}{@{}llr@{}} \toprule
  \multicolumn{2}{c}{Item} \\ \cmidrule(r){1-2}
  Animal & Description & Price (\$) \\ \midrule
  Gnat  & per gram  & 13.65 \\
        & each    & 50.00 \\
  Gnu   & stuffed & 10.00 \\
  Emu   & stuffed   & 5.00 \\
  \bottomrule
  \end{tabular}
  \caption{An example table}
\end{table}

\bibliography{example}
\bibliographystyle{plainnat}


\end{document}
